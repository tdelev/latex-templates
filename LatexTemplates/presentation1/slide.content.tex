%%%%%%%%%%%%%%%%%%%%%%%%%%%%%%%%%%%%%%%%%
%%%%%%%%%% Content starts here %%%%%%%%%%
%%%%%%%%%%%%%%%%%%%%%%%%%%%%%%%%%%%%%%%%%


\begin{frame}{ 
%Slide title%
}
%Slide content
%Example list
\begin{itemize}
  \item \texttt{strcpy} - копирање на една текстуална низа во друга
  \item \texttt{strncpy} - копирање на n бајти во тесктуална низа, се
  копираат од \texttt{src} или се додаваат \texttt{nulls}
  \item \texttt{strcat} - додава една текстуална низа на крајот на друга
  \item \texttt{strncat} - додава n бајти од една текстуална низа во друга
\end{itemize}
\end{frame}


\begin{frame}[fragile]{
%Example slide with source code
}{
%Subtitle
}
\begin{lstlisting}
#include <stdio.h>
#include <ctype.h>
#include <string.h>
#define MAX 100

void trim(char *s) {
    char *d = s;
    while (isspace(*s++))
        ;
    s--;
    while (*d++ = *s++)
        ;
    d--;
    while (isspace(*--d))
        *d = 0;
}
int main() {
    char s[MAX];
    printf("Vnesete string: ");
    gets(s);
    printf("[%s] -> ", s);
    trim(s);
    printf("[%s]", s);
    return 0;
}
\end{lstlisting}
\end{frame}
