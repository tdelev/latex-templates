%%%%%%%%%%%%%%%%%%%%%%%%%%%%%%%%%%%%%%%%%
%%%%%%%%%% Content starts here %%%%%%%%%%
%%%%%%%%%%%%%%%%%%%%%%%%%%%%%%%%%%%%%%%%%
\section{Динамичка алокација на меморија}

\begin{frame}{Задача 1}{Класа за динамичко поле}
Да се напише класа за работа со еднодимензионални полиња од целобројни елементи.
За полето се чуваат информации за вкупниот капцитет на полето, тековниот број на
елементи. Резервацијата на меморијата да се врши динамички. 
Да се преоптоварат следните оператори 
\begin{itemize}
  \item \texttt{[]} за пристап до елемент и промена на вредноста на
  елемент од полето
  \item \texttt{+=} за додавање нови броеви во полето и притоа ако е исполнет
  капацитетот на полето да се зголеми за 100\%.
\end{itemize}
Да се напише главна програма каде ќе се инстанцира објект од класата и во него
ќе се внесат N броеви од тастатура и потоа да се испечатат елементите на полето,
неговиот капацитет и вкупниот број на елементи.
\end{frame}


\begin{frame}{Задача 3}{Композиција}
Да се напише класа \texttt{Datum}, во која ќе се чуваат ден, месец и година (цели
броеви).\\
Да се напише класа \texttt{Vraboten}, за кој се чува име (не повеќе од 100
знаци) и датум на раѓање (објект од Datum).\\ 
Да се напише класа Firma, во која се чува име на фирмата (не повеќе од 100
знаци) и низа од вработени (динамичи алоцирана низа од објекти од Vraboten).
Да се преоптовари операторот += за додавање на вработен во фирмата.
За оваа класа да се имплементираат метод кој ќе ги печати
сите вработени во фирмата и метод кој ќе го пронајде и испечати најмладиот вработен.
\end{frame}

\begin{frame}[fragile]{Задача 3}{Решение 1/5}
%\lstinputlisting[lastline=25]{src/av4/z3.cpp}
\end{frame}

